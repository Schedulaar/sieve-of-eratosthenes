
\documentclass[a4paper,12pt]{article} 
\usepackage[top = 2.5cm, bottom = 2.5cm, left = 2.5cm, right = 2.5cm]{geometry} 

\usepackage[T1]{fontenc}
\usepackage[utf8]{inputenc}

\usepackage{multirow} 
\usepackage{booktabs}
\usepackage{graphicx} 


\usepackage{setspace}
\setlength{\parindent}{0in}

\usepackage{float}

\usepackage{fancyhdr}
\pagestyle{fancy}
\fancyhf{}

\usepackage{mathtools}
\DeclarePairedDelimiter\ceil{\lceil}{\rceil}
\DeclarePairedDelimiter\floor{\lfloor}{\rfloor}

\lhead{\footnotesize The Sieve of Eratosthenes}
\rhead{\footnotesize Michael Markl} 


\cfoot{\footnotesize \thepage} 

\begin{document}

\thispagestyle{empty}

\begin{tabular}{p{15.5cm}} %
{\large \bf Parallel Algorithms} \\
University of Utrecht \\ Fall 2020  \\ Rob H. Bisseling\\
\hline
\\
\end{tabular}

\vspace*{0.3cm}
\begin{center}
	{\Large \bf The Sieve of Eratosthenes}
	\vspace{2mm}
	
	{\bf Michael Markl}
\end{center}  
\vspace{0.4cm}

Prime numbers play a central role for example in modern encryption algorithms.
Most encryption algorithms need prime numbers with about 1,000 digits as a secret key.
Hence, many algorithms chose some number randomly and then check with so-called primality tests, whether this number is a prime number and therefore suitable for using as an encryption key.

One such primality test, which was already invented by the ancient greeks, is called the Sieve of Eratosthenes.
This algorithm not only checks for a given number whether it is prime, but it even finds all prime numbers up to that number.
Given the number $n$ the procedure starts with a list of the numbers from $2$ to $n$.
Start with $2$ and cross out all larger multiples of $2$.
Then continue with the next number which is not crossed out  -- in the second iteration the number $3$ -- and cross out all larger multiples of it.
Repeat this until you have worked through the whole list.
Then all numbers that are not crossed out in the end are primes.

For the primality test of $n$ it even suffices to iterate this procedure until the number~$\floor{\sqrt{n}}$: If we assume, that $n$ is not a prime number, we find numbers $a,b\in\{2, \dots, n\}$ with $n=a\cdot b$.
If $a$ or $b$ is smaller than or equal to $\floor{\sqrt{n}}$ then $n$ would be crossed out after reaching $\floor{\sqrt{n}}$.
Otherwise, $a$ and $b$ are strictly larger than $\floor{\sqrt{n}}$ which implies that $a$ and $b$ are strictly larger than $\sqrt{n}$ and hence $a\cdot b > \sqrt{n}^2 = n$ forms a contradiction.



\end{document}